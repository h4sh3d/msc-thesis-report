\chapter{Introduction}
\label{chap:introduction}

What is Bitcoin? why do we need it? \cite{crypto-2001-1592, DBLP:conf/acns/GennaroGN16, Paillier:1999:PCB:1756123.1756146,
Goldfeder2015SecuringBW, Bellare:1993:ROP:168588.168596}

ECDSA is the signature scheme used by Bitcoin to sign transactions. A standard transaction
is constituted of a single signature corresponding to the address where the Bitcoins come
from. But sometimes we need more complex management for locking funds. To address the
limitation of a single signature, Bitcoin introduced a new OP\_CODE named CHECKMULTISIG with
a new standard script. With this standard script, it is now possible to spend Bitcoin to an
address that requires a minimum of m signatures in n authorised signatories and extend the
capability of Bitcoin to lock funds in a more complex way.

However, some issues appear. The way the script works requires exposing all the public keys
when an output is signed and this increases the transaction size enormously, which implies
bigger fees. All the signatures are, obviously, present with the public keys in the transaction
script, which implies that we can know which public keys signed the transaction. And there
is some limitation, due to the script size limit, the maximum number of authorised signatories
is 15. All these limitations mean that we cannot imagine a complex organization nor structure
with the multi-signature script for the moment.

To address this limitation, a group of researchers published a first paper  in 2015 and a
second one  in 2016 describing the way to achieve a threshold scheme with DSA and ECDSA.
Today, there is no well-known implementation ready for production purposes even though
industries need it. The principal purpose of this thesis is to provide a clear, well
documented C library, based on the internal ECDSA library present in bitcoin-core.

The largest challenge in Bitcoin for the coming years is scalability. Currently, Bitcoin enforces
a block-size limit which is equivalent to only some transactions per second on the network. This
is not sufficient in comparison to big payment infrastructures such as VISA, which allows tens of
thousands of transactions per second and even more in peak times such as Christmas. To address this,
some proposals modifying the transaction structure (like SegWit), some proposals modifying the
block-size limit (such as SegWit2x) and others creating a second layer based on top of the Bitcoin
protocol (like Lightning Network) exist. In the same idea of the Lightning Network, Bity is working
on an implementation of a one-way payment channel. A one-way payment channel allows two parties to
transact over the blockchain while minimizing the number of transactions needed on the blockchain
in a secure and trustless way. This kind of channel needs multi-signature addresses which might be
improved with the threshold scheme. The second part of the thesis is to co-write the channel white
paper and add a chapter of how to improve it with the threshold scheme (better privacy, cheaper
transaction, less limitations).
