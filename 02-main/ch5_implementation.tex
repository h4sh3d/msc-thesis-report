\chapter{Implementation in Bitcoin-core secp256k1}
\label{chap:implementation-secp256k1}

As previously mentioned, Bitcoin uses \gls{ecc} for signing
transactions. When the first release of Bitcoin-core appeared in early 2009,
OpenSSL library was used to perform the cryptographic computations.  Several years
later, a project started with the goal of replacing OpenSSL and creating a custom
and minimalistic \texttt{C} library for cryptography over the \texttt{secp256k1} curve.
This library is now available on GitHub at \texttt{bitcoin-core/secp256k1}
and it is one of the most optimized libraries, if not the most,
for this curve. It is worth noting that other significant
crypto-currencies like Ethereum also use this library, so extending the
capabilities of this library is an excellent choice to attract other
cryptographers to have a look and increase the number of reviews for this
thesis.

The implementation is spread into four main components (i) a \texttt{DER}
parser-serializer, (ii) a naïve implementation of Paillier homomorphic
cryptosystem, (iii) implementation of the adapted Zero-Knowledge Proofs, and
(iv) the threshold public API. Noted that the current
implementation is NOT production ready and NOT side-channel attack resistant.
Paillier and ZKP are not constant time computation and use \texttt{libgmp} for
all arithmetic computations, even when secret values are used. This
implementation is a textbook implementation of the scheme and needs to be
reviewed and tested more thoroughly before being used in production. It is also worth
noting that this library does not implement the functions needed to initialize
the setup. Only the functions needed to parse existing keys and compute a
distributed signature are implemented.

This chapter refers to the implementation available on GitHub at
\url{https://github.com/GuggerJoel/secp256k1/tree/threshold} at the time of
writing. The sources may evolve after writing this report, to be sure to read
the latest version of the code check out the sources directly on GitHub.

\minitoc

\newpage

% -----------------------------------------------------------------------------
\section{Configuration}

The library uses \texttt{autotools} to manage the compilation, installation, and
uninstallation. A system of modules already existing in the structure with an
ECDH experimental module for shared secret computation and a recovery module for
recovering public keys from signatures. Modules can be flagged as experimental, then, at
configuration time, an explicit parameter enabling experimental modules must be
passed, and a warning is shown to warn that the build contains experimental
code.

\subsection{Add new experimental module}

In this structure, the threshold extension is all indicated to be an
experimental module also. A new variable \texttt{\$enable\_module\_recovery} is
declared with an \texttt{m4} macro defined by \textit{autoconf} in the \texttt{configure.ac} file
with the argument \texttt{-{}-enable-module-threshold}. The default value is set
to \texttt{no} to not enable the module by default.

\begin{listing}
  \bashfile[firstline=137,lastline=140]{02-main/listings/configure.ac}
	\caption{Add argument in \texttt{configure.ac} to enable the module}
	\label{lst:configureEnableThreshold}
\end{listing}

If the variable \texttt{\$enable\_module\_recovery} is set to \texttt{yes} in
\texttt{configure.ac} (lines 443 to 445), a compiler constant is declared, again
with an \texttt{m4} marco defined by \textit{autoconf}, and set to \texttt{1} in
\texttt{libsecp256k1-config.h} (lines 20 and 21). This header file is generated
when \texttt{./configure} script runs and is included in the library.

\begin{listing}
  \bashfile[firstline=443,lastline=445]{02-main/listings/configure.ac}
  \cfile[firstline=20,lastline=21]{02-main/listings/libsecp256k1-config.h}
	\caption{Define constant \texttt{ENABLE\_MODULE\_THRESHOLD} if module enable}
	\label{lst:defineEnableThreshold}
\end{listing}

The main file \texttt{secp256k1.c} (lines 586 to 590) and the tests file
\texttt{tests.c} include headers based on the compiler constant definition.

\begin{listing}
  \cfile[firstline=586,lastline=590]{02-main/listings/secp256k1.c}
	\caption{Including implementation headers if \texttt{ENABLE\_MODULE\_THRESHOLD} is
  defined}
	\label{lst:includeThresholdImplementationHeaders}
\end{listing}

The module is set as experimental to avoid enabling it without explicitly agreeing
to build experimental code. If the experimental parameter is set to \texttt{yes} a
warning is displayed during the configuration process to warn the user. If the
experimental parameter is not set and an experimental module is enabled an error
message is displayed, and the process fails.

\begin{listing}
  \bashfile[firstline=465,lastline=482]{02-main/listings/configure.ac}
	\caption{Set threshold module to experimental in \texttt{configure.ac}}
	\label{lst:setModuleExperimental}
\end{listing}

\subsection{Configure compilation}

A module is composed of one or many \texttt{include/} headers that contain the
public API with a small description of each function, these headers are copied
in the right folders when \texttt{sudo make install} command is run. The file
\texttt{Makefile.am} defines which headers need to be installed, which do not and
how to compile the project. This file is parsed by \textit{autoconf} to generate the
final \texttt{Makefile}.

Each module has its \texttt{Makefile.am.include} which describes what to do with
all the files in the module folder. This file is included in the main
\texttt{Makefile.am} (lines 179 to 181) if the module is enabled.

\begin{listing}
  \makefile[firstline=179,lastline=181]{02-main/listings/Makefile.am}
	\caption{Include specialized Makefile if threshold module is enabled}
	\label{lst:includeSpecializedMakefile}
\end{listing}

The specific \texttt{Makefile.am.include} declares the requisite header to be
included and declares the list of all the headers that must not be installed on
the system when \texttt{sudo make install} command is run.

\begin{listing}
  \makefile{02-main/listings/Makefile.am.include}
	\caption{Specialized Makefile for threshold module}
	\label{lst:specializedMakefile}
\end{listing}

It is possible to build the library and enable the threshold module with the
command

\begin{minted}[breaklines=true,fontsize=\scriptsize]{bash}
./configure --enable-module-threshold --enable-experimental
\end{minted}

% -----------------------------------------------------------------------------

\section{\texttt{DER} parser-serializer}

Transmit messages and retrieve keys are an essential part of the scheme. Because
between each step communication on the network is necessary, a way to export
and import data is required. Bitcoin private keys are a simple structure because
of the fixed curve and their intrinsic nature, a single $2^{256}$ bits value.
Threshold private keys are composed of multiple parts (i) a private
share, (ii) a Paillier private key, (iii) a Paillier public key, and (iv)
Zero-Knowledge Proof parameters. To serialize these complex structures the
\texttt{DER} standard is chosen. Three simple data types are implemented in the
library (i) sequence, (ii) integer, and (iii) octet string.

\subsection{Sequence}

The sequence data structure holds a sequence of other data types.
The sequence starts with the constant byte \texttt{0x30} and is followed by the
content length and then the content itself. A length could be in the short form or
the long form. If the content number of bytes is shorter than \texttt{0x80} the
length byte indicates the length, if the content is equal or longer than
\texttt{0x80} the seven lower bits 0 to 6 where $\texttt{byte} = \{
\texttt{b}_7, \dots, \texttt{b}_1, \texttt{b}_0\}$ indicate the number of
following bytes which are used for the length.

\begin{listing}
  \cfile[firstline=10,lastline=23]{02-main/listings/der_impl.h}
	\caption{Implementation of a \texttt{DER} length parser}
	\label{lst:implDERLengthParser}
\end{listing}

The sequence parser checks the first byte with the constant byte \texttt{0x30} and
extracts the content length. Positions in the input array are held in the
\texttt{*pos} variable, extracted length is stored in \texttt{*length}, and the
offset holds how many bytes of the data are used for the header and the length.
A coherence check is performed to ensure that the current offset and the
retrieved length result in the same number of bytes passed in argument.

When a sequence holds other sequences, retrieving their total length (including
header and content length bytes) is needed to parse them recursively. A specific
function is created to retrieve the total length of a struct given a pointer to
its first byte.

\begin{listing}
  \cfile[firstline=25,lastline=35]{02-main/listings/der_impl.h}
	\caption{Implementation of a \texttt{DER} sequence parser}
	\label{lst:implDERSequenceParser}
\end{listing}

The serialization of a sequence is implemented as a serialization of an octet
string with the sequence byte \texttt{0x30} without an integrity check of the
content. The content length is serialized first, then the constant byte is added.

The result of content length serialization can be $\geq 1$ bytes. If the
content is shorter than \texttt{0x80}, then one byte is enough to store the
length. Otherwise multiple bytes ($\geq 2$) are used. Because the number of bytes is
undefined before the computation, a memory allocation is necessary and a pointer
is returned with the length of the array.

\begin{listing}
  \cfile[firstline=155,lastline=166]{02-main/listings/der_impl.h}
	\caption{Implementation of a \texttt{DER} sequence serializer}
	\label{lst:implDERSequenceSerializer}
\end{listing}

If the content length is longer than \texttt{0x80}, then \texttt{libgmp} is used to
serialize the length into a bytes array in big-endian most significant byte
first. The length of this serialization is stored in \texttt{longsize} and is
used to create the first byte with the most significant bit set to 1 (line 93).

\begin{listing}
  \cfile[firstline=81,lastline=100]{02-main/listings/der_impl.h}
	\caption{Implementation of a \texttt{DER} length serializer}
	\label{lst:implDERLengthSerializer}
\end{listing}

\subsection{Integer}

Integers are used to store most values in the keys and Zero-Knowledge
Proofs. An integer can be positive, negative or zero and are represented in the
second complement form. The byte header starts with \texttt{0x02}, followed by the
length of the data. Parsing and serializing integers is already implemented in
\texttt{libgmp}, functions are just wrappers to extract information from the
header and start the \texttt{mpz} importation at the right offset.

\subsection{Octet string}

Octet strings are used to hold serialized data like points or public keys. An
octet string is an arbitrary array of bytes. The header starts with the byte header
\texttt{0x04} followed by the size of the content. The serialization
implementation retrieves the length of the content, copies the header and the
octet string into a new memory space, and returns the pointer with the total
length. The parser implementation copies the content and sets the content length,
the position index, and the offset.

\section{Paillier cryptosystem}

Homomorphic encryption is required in the scheme and Paillier is proposed in the
white paper. Paillier homomorphic encryption is simple to implement in a
naïve way. This implementation is functional but not optimized and needs to
be reviewed.

\subsection{Data structures}

Encrypted messages, public keys and private keys are transmitted. As mentioned before,
the \texttt{DER} standard format is used to parse and serialize data. A
\texttt{DER} schema for all data structures is defined to ensure portability
over different implementations.

\subsubsection{Public keys}

The public key is composed of a public modulus and a generator. The
implementation data structure adds a big modulus corresponding to the square of
the modulus. A version number is added for future compatibility purposes.

\begin{listing}
  \begin{minted}[breaklines=true,fontsize=\scriptsize]{text}
HEPublicKey ::= SEQUENCE {
    version           INTEGER,
    modulus           INTEGER,  -- p * q
    generator         INTEGER
}
  \end{minted}
	\caption{\texttt{DER} schema of a Paillier public key}
	\label{lst:DERSchemaPaillierPub}
\end{listing}

\texttt{libgmp} is used for all the arithmetic in the Paillier implementation, all
numbers are stored in \texttt{mpz\_t} type. The parser takes as input an array
of bytes with a length and the public key to create.

\begin{listing}
  \begin{minted}[breaklines=true,fontsize=\scriptsize]{c}
typedef struct {
    mpz_t modulus;
    mpz_t generator;
    mpz_t bigModulus;
} secp256k1_paillier_pubkey;

int secp256k1_paillier_pubkey_parse(
    secp256k1_paillier_pubkey *pubkey,
    const unsigned char *input,
    size_t inputlen
);
  \end{minted}
	\caption{\texttt{DER} parser of a Paillier public key}
	\label{lst:DERImplPaillierParsePub}
\end{listing}


\subsubsection{Private keys}

The private key is composed of a public modulus, two primes, a generator, a
private exponent $\lambda = \varphi(n) = (p-1)(q-1)$, and a private coefficient
$\mu = \varphi(n)^{-1} \mod n$. Again, a version number is added for future
compatibility purposes.

\begin{listing}
  \begin{minted}[breaklines=true,fontsize=\scriptsize]{text}
HEPrivateKey ::= SEQUENCE {
    version           INTEGER,
    modulus           INTEGER,  -- p * q
    prime1            INTEGER,  -- p
    prime2            INTEGER,  -- q
    generator         INTEGER,
    privateExponent   INTEGER,  -- (p - 1) * (q - 1)
    coefficient       INTEGER   -- (inverse of privateExponent) mod (p * q)
}
  \end{minted}
	\caption{\texttt{DER} schema of a Paillier private key}
	\label{lst:DERSchemaPaillierPriv}
\end{listing}

The parser takes as input, an array of bytes with a length and the private key to
create. The big modulus is computed after parsing to accelerate encryption
and decryption.

\begin{listing}
  \begin{minted}[breaklines=true,fontsize=\scriptsize]{c}
typedef struct {
    mpz_t modulus;
    mpz_t prime1;
    mpz_t prime2;
    mpz_t generator;
    mpz_t bigModulus;
    mpz_t privateExponent;
    mpz_t coefficient;
} secp256k1_paillier_privkey;

int secp256k1_paillier_privkey_parse(
    secp256k1_paillier_privkey *privkey,
    secp256k1_paillier_pubkey *pubkey,
    const unsigned char *input,
    size_t inputlen
);
  \end{minted}
	\caption{\texttt{DER} parser of a Paillier private key}
	\label{lst:DERImplPaillierParsePriv}
\end{listing}

\subsubsection{Encrypted messages}

An encrypted message with Paillier cryptosystem is a big number $c \in
\mathbb{Z}_{n^2}^*$. No version number is added in this case. The implementation's
structure contains a \textit{nonce} value which could be set to 0 to store the \textit{nonce} used
during encryption.

\begin{listing}
  \begin{minted}[breaklines=true,fontsize=\scriptsize]{text}
HEEncryptedMessage ::= SEQUENCE {
    message           INTEGER
}
  \end{minted}
	\caption{\texttt{DER} schema of an encrypted message with Paillier cryptosystem}
	\label{lst:DERSchemaPaillierEncMessage}
\end{listing}

Encrypted messages can be serialized and parsed and they are used in message
exchange during the signing protocol by both signers.

\subsection{Encrypt and decrypt}

Like all other encryption schemes in public key cryptography, the public key is
used to encrypt and the private key to decrypt. To encrypt the message
\texttt{mpz\_t m} where $m < n$ a random value $r$ where $r < n$ is selected
with the function pointer \texttt{noncefp} and set in the \textit{nonce} value
\texttt{res->nonce}. This \textit{nonce} is stored because its value is needed to create the
Zero-Knowledge Proofs. The cipher $c = g^m \cdot r^n \mod n^2$ is put
in \texttt{res->message} to complete the encryption process. All intermediary
states are erased before returning the result.

\begin{listing}
  \begin{minted}[breaklines=true,fontsize=\scriptsize]{c}
int secp256k1_paillier_encrypt_mpz(secp256k1_paillier_encrypted_message *res, const mpz_t m, const secp256k1_paillier_pubkey *pubkey, const secp256k1_paillier_nonce_function noncefp) {
    mpz_t l1, l2, l3;
    int ret = noncefp(res->nonce, pubkey->modulus);
    if (ret) {
        mpz_inits(l1, l2, l3, NULL);
        mpz_powm(l1, pubkey->generator, m, pubkey->bigModulus);
        mpz_powm(l2, res->nonce, pubkey->modulus, pubkey->bigModulus);
        mpz_mul(l3, l1, l2);
        mpz_mod(res->message, l3, pubkey->bigModulus);
        mpz_clears(l1, l2, l3, NULL);
    }
    return ret;
}
  \end{minted}
	\caption{Implementation of encryption with Paillier cryptosystem}
	\label{lst:implEncryptPaillier}
\end{listing}

If the random value selection process fails the encryption also fails. The
random function of type \texttt{secp256k1\_paillier\_nonce\_function} must use a
good CPRNG and its implementation is not part of the library.

\begin{listing}
  \begin{minted}[breaklines=true,fontsize=\scriptsize]{c}
typedef int (*secp256k1_paillier_nonce_function)(
    mpz_t nonce,
    const mpz_t max
);
  \end{minted}
	\caption{Function signature for Paillier \textit{nonce} generation}
	\label{lst:PaillierNoncesSigFunc}
\end{listing}

To decrypt the cipher $c \in \mathbb{Z}_{n^2}^*$ with the private key, the
function computes $m = L(c^{\lambda} \mod n^2) \cdot \mu \mod n$ where $L(x) = (x -
1) / n$. The cipher is raised to the lambda $c^{\lambda} \mod n^2$ in line 4 and
the result is put in an intermediary state variable. Then, the $L(x)$ function is
applied to the intermediary state in lines 5-6. Finally, the multiplication with
$\mu$ is performed and the modulo $n$ is applied (lines 7-8) to lead to the result. It is
worth noting that, in line 6, only the quotient of the division is recovered.

\begin{listing}
  \begin{minted}[breaklines=true,linenos,fontsize=\scriptsize]{c}
void secp256k1_paillier_decrypt(mpz_t res, const secp256k1_paillier_encrypted_message *c, const secp256k1_paillier_privkey *privkey) {
    mpz_t l1, l2;
    mpz_inits(l1, l2, NULL);
    mpz_powm(l1, c->message, privkey->privateExponent, privkey->bigModulus);
    mpz_sub_ui(l2, l1, 1);
    mpz_cdiv_q(l1, l2, privkey->modulus);
    mpz_mul(l2, l1, privkey->coefficient);
    mpz_mod(res, l2, privkey->modulus);
    mpz_clears(l1, l2, NULL);
}
  \end{minted}
	\caption{Implementation of decryption with Paillier cryptosystem}
	\label{lst:implDecryptPaillier}
\end{listing}

\subsection{Homomorphism}

The choice of this scheme is not hazardous, homomorphic addition and
multiplication are used to construct the signature $s = D_{sk}(\mu)
\mod q : \mu = (\alpha \times_{pk} m'z_2) +_{pk} (\zeta \times_{pk} r'x_2z_2)
+_{pk} E_{pk}(cn)$ where $+_{pk}$ denotes homomorphic addition over the
ciphertexts and $\times_{pk}$ denotes homomorphic multiplication over the
ciphertexts.

\subsubsection{Addition}
Addition $+_{pk}$ over ciphertexts is computed with $D_{sk}(E_{pk}(m_1, r_1)
\cdot E_{pk}(m_2, r_2) \mod n^2) = m_1 + m_2 \mod n$ or $D_{sk}(E_{pk}(m_1, r_1)
\cdot g^{m_2} \mod n^2) = m_1 + m_2 \mod n$ where $D_{sk}$ denotes decryption
with private key $sk$ and $E_{pk}$ denotes encryption with public key $pk$. Only
the first variant is implemented, where two ciphertexts are added together to
result in a third ciphertext.

\begin{listing}
  \begin{minted}[breaklines=true,fontsize=\scriptsize]{c}
void secp256k1_paillier_add(secp256k1_paillier_encrypted_message *res, const secp256k1_paillier_encrypted_message *op1, const secp256k1_paillier_encrypted_message *op2, const secp256k1_paillier_pubkey *pubkey) {
    mpz_t l1;
    mpz_init(l1);
    mpz_mul(l1, op1->message, op2->message);
    mpz_mod(res->message, l1, pubkey->bigModulus);
    mpz_clear(l1);
}
  \end{minted}
	\caption{Implementation of homomorphic addition with Paillier cryptosystem}
	\label{lst:implHomomorphAddPaillier}
\end{listing}

\subsubsection{Multiplication}

Multiplication $\times_{pk}$ over ciphertexts can be performed with
$D_{sk}(E_{pk}(m_1, r_1)^{m_2} \mod n^2) = m_1 m_2 \mod n$, the implementation
is straight forward in this case. The \textit{nonce} value from the ciphertext is copied
in the resulting encrypted message to not lose information after opperations.

\begin{listing}
  \begin{minted}[breaklines=true,fontsize=\scriptsize]{c}
void secp256k1_paillier_mult(secp256k1_paillier_encrypted_message *res, const secp256k1_paillier_encrypted_message *c, const mpz_t s, const secp256k1_paillier_pubkey *pubkey) {
    mpz_powm(res->message, c->message, s, pubkey->bigModulus);
    mpz_set(res->nonce, c->nonce);
}
  \end{minted}
	\caption{Implementation of homomorphic multiplication with Paillier cryptosystem}
	\label{lst:implHomomorphMulPaillier}
\end{listing}

% -----------------------------------------------------------------------------

\section{Zero-knowledge proofs}

Two Zero-Knowledge Proofs are used in the scheme, each party generates a proof
and validates the other one. A proof is generated and verified under some ZKP
parameters, these parameters are fixed at initialization time and don't
change over time.

\subsection{Data structures}

Three data structures are created, one for each ZKP and one for storing the
parameters. Zero-Knowledge Proofs are composed of big numbers and points, and
need to be serialized and parsed to be included in the message exchange
protocol.

\subsubsection{Zero-Knowledge Parameters}

Zero-Knowledge parameter is composed of three numeric values (i) $\tilde{N}$ a
public modulus, (ii) a value $h_2$ selected randomly $h_2 \in
\mathbb{Z}_{\tilde{N}}^*$, and (iii) a value $h_1$ where $\exists x, \log_x(h_1) =
h_2 \mod \tilde{N}$. Only the function to parse a \texttt{ZKPParameter}
\texttt{DER} schema is provided in the module.

\begin{listing}
  \begin{minted}[breaklines=true,fontsize=\scriptsize]{text}
ZKPParameter ::= SEQUENCE {
    modulus            INTEGER,
    h1                 INTEGER,
    h2                 INTEGER
}
  \end{minted}
	\caption{\texttt{DER} schema of a Zero-Knowledge parameters sequence}
	\label{lst:DERSchemaZKPParams}
\end{listing}

\subsubsection{Zero-Knowledge Proof $\Pi$}

Zero-Knowledge Proof $\Pi$ is composed of numeric values and one point. The
point is stored in a public key internal structure within the implementation and
is exported through the \texttt{secp256k1} library as a 65 byte uncompressed public key.
The uncompressed public key is then stored as an octet string in the schema. A
version number is added for future compatibility purposes. Two functions are
provided in the module to parse and serialize a \texttt{ECZKPPi} \texttt{DER}
schema.

\begin{listing}
  \begin{minted}[breaklines=true,fontsize=\scriptsize]{text}
ECZKPPi ::= SEQUENCE {
    version            INTEGER,
    z1                 INTEGER,
    z2                 INTEGER,
    y                  OCTET STRING,
    e                  INTEGER,
    s1                 INTEGER,
    s2                 INTEGER,
    s3                 INTEGER,
    t1                 INTEGER,
    t2                 INTEGER,
    t3                 INTEGER,
    t4                 INTEGER
}
  \end{minted}
	\caption{\texttt{DER} schema of a Zero-Knowledge $\Pi$ sequence}
	\label{lst:DERSchemaZKPPi}
\end{listing}

\subsubsection{Zero-Knowledge Proof $\Pi'$}

Zero-Knowledge Proof $\Pi'$ is composed of the same named values as ZKP $\Pi$
plus five new ones. The construction of the proof is based on $\Pi$ but needs
more equations to express all the proven statements. Again, the point $y$ is a
point serialized as an uncompressed public key in an octet string and a version
number is added for future compatibility purposes. Two functions are provided in
the module to parse and serialize a \texttt{ECZKPPiPrim} \texttt{DER} schema.

\begin{listing}
  \begin{minted}[breaklines=true,fontsize=\scriptsize]{text}
ECZKPPiPrim ::= SEQUENCE {
    version            INTEGER,
    z1                 INTEGER,
    z2                 INTEGER,
    z3                 INTEGER,
    y                  OCTET STRING,
    e                  INTEGER,
    s1                 INTEGER,
    s2                 INTEGER,
    s3                 INTEGER,
    s4                 INTEGER,
    t1                 INTEGER,
    t2                 INTEGER,
    t3                 INTEGER,
    t4                 INTEGER,
    t5                 INTEGER,
    t6                 INTEGER,
    t7                 INTEGER
}
  \end{minted}
	\caption{\texttt{DER} schema of a Zero-Knowledge $\Pi'$ sequence}
	\label{lst:DERSchemaZKPPi2}
\end{listing}

\subsection{Generate proofs}

Proofs are generated in relation to a specific setup and a specific \say{in progress
signature}, which makes them linked to a large number of values (points,
encrypted messages, secrets, parameters, etc.) The complexity of these
constructions is strongly felt in the code. Heavy mathematical computations are
needed with two \texttt{hash} functions.

A CPRNG function is required to generate both proofs. This function generates
random numbers in $\mathbb{Z}_{max}$ and $\mathbb{Z}_{max}^*$. The \texttt{flag}
argument indicates which case is treated, \texttt{STD} or \texttt{INV}. If the
function has no access to a good source of randomness or cannot generate a good
random number a zero is returned, otherwise a one is returned.

\begin{listing}
  \begin{minted}[breaklines=true,fontsize=\scriptsize]{c}
typedef int (*secp256k1_eczkp_rdn_function)(
    mpz_t res,
    const mpz_t max,
    const int flag
);

#define SECP256K1_THRESHOLD_RND_INV 0x01
#define SECP256K1_THRESHOLD_RND_STD 0x00
  \end{minted}
	\caption{Function signature for ZKP CPRNG}
	\label{lst:funcSigZKPCPRGN}
\end{listing}

\subsubsection{Zero-Knowledge Proof $\Pi$}

As shown in Figure~\ref{fig:theProofPi}, the proof states that (i) there exists a
known value by the proover that links $R \rightarrow R_2$, (ii) there exists a
second known value by the proover that, related to the first one, links $G
\rightarrow Y_1$, (iii) the result of $D_{sk}(\alpha)$ is this first value, and
(iv) the result of $D_{sk}(\zeta)$ is this second value.

To do computation on the curve a context object needs to be passed in arguments,
then the ZKP object to create, the ZKP parameters, the two encrypted messages
$\alpha$ and $\zeta$, scalar values $sx_1$ and $sx_2$ representing $z_1 =
(k_1)^{-1} \mod n$ and $x_1z_1$, then the point $R$, the point $R_2$, the
partial public key $Y_1$, the proover Paillier public key which has been used to
encrypt $\alpha$ and $\zeta$, and finally a pointer to a CPRNG function used to
generate all needed random values.

\begin{listing}
  \begin{minted}[breaklines=true,fontsize=\scriptsize]{c}
int secp256k1_eczkp_pi_generate(
    const secp256k1_context *ctx,
    secp256k1_eczkp_pi *pi,
    const secp256k1_eczkp_parameter *zkp,
    const secp256k1_paillier_encrypted_message *m1,
    const secp256k1_paillier_encrypted_message *m2,
    const secp256k1_scalar *sx1,
    const secp256k1_scalar *sx2,
    const secp256k1_pubkey *c,
    const secp256k1_pubkey *w1,
    const secp256k1_pubkey *w2,
    const secp256k1_paillier_pubkey *pubkey,
    const secp256k1_eczkp_rdn_function rdnfp
);
  \end{minted}
	\caption{Function signature to generate ZKP $\Pi$}
	\label{lst:funcSigGeneratePi}
\end{listing}

The function's implementation can be split in four main parts (i) generate all
the required random values, (ii) compute the proof values $v$, (iii) compute
the challenge value with the \texttt{hash} of these values $v$, and (iv) compute
the ZKP with $e = \texttt{hash}(v)$.

\subsubsection{Zero-Knowledge Proof $\Pi'$}

As shown in Figure~\ref{fig:theProofPi2}, the proof states that (i) there exists a
known value by the proover $x_1$ that link $R_2 \rightarrow G$, (ii) there exists a
second known value by the proover that, related to the first one, link $G
\rightarrow Y_2$, (iii) the result of $D_{sk'}(\mu')$ is this first value, and
(iv) there exists a third known value by the proover $x_3$ and the result of
$D_{sk}(\mu)$ is the homomorphic operation of $(\alpha \times x_1) + (\zeta
\times x_2) + x_3$.

\begin{listing}
  \begin{minted}[breaklines=true,fontsize=\scriptsize]{c}
int secp256k1_eczkp_pi2_generate(
    const secp256k1_context *ctx,
    secp256k1_eczkp_pi2 *pi2,
    const secp256k1_eczkp_parameter *zkp,
    const secp256k1_paillier_encrypted_message *m1,
    const secp256k1_paillier_encrypted_message *m2,
    const secp256k1_paillier_encrypted_message *m3,
    const secp256k1_paillier_encrypted_message *m4,
    const secp256k1_paillier_encrypted_message *r,
    const mpz_t x1,
    const mpz_t x2,
    const mpz_t x3,
    const mpz_t x4,
    const mpz_t x5,
    const secp256k1_pubkey *c,
    const secp256k1_pubkey *w2,
    const secp256k1_paillier_pubkey *pairedkey,
    const secp256k1_paillier_pubkey *pubkey,
    const secp256k1_eczkp_rdn_function rdnfp
);
  \end{minted}
	\caption{Function signature to generate ZKP $\Pi'$}
	\label{lst:funcSigGeneratePi2}
\end{listing}

The function's implementation can also be split in four main parts (i) generate
all the required random values, (ii) compute the proof values $v$, (iii) compute
the challenge value with the \texttt{hash'} of these values $v$, and (iv)
compute the ZKP with $e = \texttt{hash'}(v)$.

As shown in the original paper \texttt{hash} and \texttt{hash'} must be different
hashing functions to avoid reusing $\Pi$ proofs, even without satisfying the
predicate, to construct fraudulent $\Pi'$ proofs.

\subsection{Validate proofs}

Validation of proofs $\Pi$ and $\Pi'$ can be done with (i) the Paillier public
keys, (ii) the ZKP parameters, and (iii) the exchanged messages. The process can
be split into three steps (a) compute the proof values, (b) retrieve the candidate
value $e'$, (c) and compare if $e = e'$. If the values match the proof is valid.

\begin{longlisting}
  \begin{minted}[breaklines=true,fontsize=\scriptsize]{c}
int secp256k1_eczkp_pi_verify(
    const secp256k1_context *ctx,
    secp256k1_eczkp_pi *pi,
    const secp256k1_eczkp_parameter *zkp,
    const secp256k1_paillier_encrypted_message *m1,
    const secp256k1_paillier_encrypted_message *m2,
    const secp256k1_pubkey *c,
    const secp256k1_pubkey *w1,
    const secp256k1_pubkey *w2,
    const secp256k1_paillier_pubkey *pubkey
);

int secp256k1_eczkp_pi2_verify(
    const secp256k1_context *ctx,
    secp256k1_eczkp_pi2 *pi2,
    const secp256k1_eczkp_parameter *zkp,
    const secp256k1_paillier_encrypted_message *m1,
    const secp256k1_paillier_encrypted_message *m2,
    const secp256k1_paillier_encrypted_message *m3,
    const secp256k1_paillier_encrypted_message *m4,
    const secp256k1_pubkey *c,
    const secp256k1_pubkey *w2,
    const secp256k1_paillier_pubkey *pubkey,
    const secp256k1_paillier_pubkey *pairedkey
);
  \end{minted}
	\caption{Function signature to validate ZKP $\Pi$ and $\Pi'$}
	\label{lst:funcSigValidatePiPi2}
\end{longlisting}

% -----------------------------------------------------------------------------

\section{Threshold module}

The threshold module exposes the public API used to create an application that
wants to use the distributed signature protocol. The public API includes all
functions needed to parse-serialize keys, messages, and signature parameters.
Signature parameters hold the values $k$, $z = k^{-1}$, and $R = k \cdot G$,
these values are---in a normal signature mode---computed, used, and destroyed in
one go. However, a mechanism to save and restore these values is required in
distributed mode because the context can be destroyed and re-created between
each step.

The public API also includes the five functions that implement the protocol. One
function is one step in the protocol or an in-between functions. The generated
messages are serialized by the caller and parsed by the sender. The signature
parameters could also be serialized and parsed during the waiting time.

\subsubsection{Nomenclature}

A proposal for exchanged message names and actions is done in this thesis.
Players $P_1$ and $P_2$ represent the initiator and collaborator. Player $P_1$
initializes the communication and asks $P_2$ to collaborate on a signature, if
$P_2$ collaborates and the protocol ends successfully $P_1$ retrieves the
signature.

Four messages are necessary between the five steps. In order, the proposed names
are (i) call message, (ii) challenge message, (iii) response challenge, and (iv)
terminate message. The functions are named after the corresponding action and
message name.

\subsection{Create call message}

The \texttt{call\_create} function, as indicated by its name, creates the call
message. Arguments are checked to be non-null.  If one of them is the function
will fail. The secret share is loaded in a 32 byte array and the \textit{nonce} ($k$) is
retrieved with the \texttt{noncefp} function pointer. It is worth noting that
this function could be called multiple times until a \textit{nonce} that is not zero and
which does not overflow is found. However, this function has a limited number of
calls and if the limit is reached the function will fail. The signature
parameters are then set and encrypted in the call message. The parameters $k$
and $z$ are set for $P_1$. The \texttt{noncefp} can point to an implementation
of a deterministic signature mode or a random signature mode. If
the deterministic model is chosen, the counter indicates the number of rounds
done by the function \cite{rfc6979}.

\begin{longlisting}
  \cfile[firstline=247,lastline=282]{02-main/listings/threshold_impl.h}
	\caption{Implementation of \texttt{call\_create} function}
	\label{lst:implCallCreateFunc}
\end{longlisting}

\subsection{Receive call message}

The \texttt{call\_received} function sets the parameter $k$ and $R$ of $P_2$ and
prepares the challenge message with $R$. Again, the pointer can point to a
deterministic implementation for generating the nonce.

\begin{longlisting}
  \cfile[firstline=284,lastline=315]{02-main/listings/threshold_impl.h}
	\caption{Implementation of \texttt{call\_received} function}
	\label{lst:implCallReceivedFunc}
\end{longlisting}

\subsection{Receive challenge message}

The \texttt{challenge\_received} function is called by $P_1$ to compute the
final public point $R$ of the signature and create the first Zero-Knowledge
Proof.

\begin{longlisting}
  \cfile[firstline=317,lastline=352]{02-main/listings/threshold_impl.h}
	\caption{Implementation of \texttt{challenge\_received} function}
	\label{lst:implChallengeReceivedFunc}
\end{longlisting}

\subsection{Receive response challenge message}

The \texttt{response\_challenge\_received} function is called by $P_2$ and
validates the first Zero-Knowledge Proof, $\Pi$. The final ciphertext which
contains the $s$ part of the distributed signature is computed and the second
Zero-Knowledge Proof $\Pi'$ is created.

The point $R$ is normalized and the coordinate $x$ of $R$ is retrieved (modulo $n$). The
\texttt{hash} is multiplied with $z_2$ and the coordinate $x$ of $R$ is multiplied
with $x_2z_2$. A value $x_3$ where $n|x_3$ ($n$ divides $x_3$) is added to the cipher to hide
information about the secret share and the secret random. In ECDSA $s = k^{-1}(m +
rx) \mod n$, so the ciphertext matches the requirement as demonstrated below:

\begin{equation}
\begin{split}
  D_{sk}(\mu) &\equiv (\alpha \times mz_2) + (\zeta \times rx_2z_2) + (x_3) \pmod n \\
              &\equiv (z_1 \times mz_2) + (x_1z_1 \times rx_2z_2)  \\
              &\equiv (z_1z_2m) + (x_1z_1rx_2z_2)  \\
              &\equiv z_1z_2(m + rx_1x_2)  \\
              &\equiv z(m + rx)  \\
              &\equiv k^{-1}(m + rx)
\end{split}
\end{equation}

\begin{longlisting}
  \cfile[firstline=379,lastline=446]{02-main/listings/threshold_impl.h}
	\caption{Core function of \texttt{response\_challenge\_received}}
	\label{lst:implRespChallengeReceivedFunc}
\end{longlisting}

\subsection{Receive terminate message}

The \texttt{terminate\_received} function is called by $P_1$ and validates the
second Zero-Knowledge Proof, $\Pi'$. After validation of the proof, the
ciphertext is decrypted and the signature is composed. The signature is then
tested and the protocol ends. Only $P_1$ can decrypt the ciphertext so the
protocol is asymmetric. If $P_2$ also needs the signature, $P_1$ must share it.
There is no way for $P_2$ to know the signature without a cooperative $P_1$.

\begin{longlisting}
  \cfile[firstline=460,lastline=526]{02-main/listings/threshold_impl.h}
	\caption{Core function of \texttt{terminate\_received}}
	\label{lst:implTerminateReceivedFunc}
\end{longlisting}
