\chapter{Conclusions}
\label{chap:conclusions}

% -----------------------------------------------------------------------------
\section{Les innovations apportées par Levee}

L'innovation majeur présente au sein de \gls{cpi} est l'approche avec laquelle
les chercheurs ont abordés la problématique de protection du flot de contrôle.
Comme montré, il existe déjà des mécanismes tels que SoftBound+CETS,
permettant de se prémunir à 100\% contre les \og control-flow hijack \fg.
Ces mécanismes appliquent les principes des langages \og memory safe \fg sur
l'entièreté du programme, alors que l'approche de \gls{cpi} sélectionne une
partie réduite des pointeurs responsables du flot de contrôle du programme.
Les résultats obtenu, tant en terme de performance qu'en terme d'efficacité,
sont très bons.

Des améliorations telles que HardBound ou Watchdog réduisant le coût en
performance de SoftBound existent. Ces améliorations repose sur des bases
hardware, permettant d'instrumenter la protection de la manière la plus efficace
possible et de soulager la couche software. Pour \gls{levee}, il en va de même,
il est possible d'augmenter les performances en se basant sur des implémentation
hardware tel que, par exemple, Intel MPX (Memory Protection Extensions)
\cite{IntelMPX}, apparu dans l'architecture Skylake, supportée au niveau
du noyau Linux depuis la version 3.19. Il est donc très intéressant de constater
qu'en plus des performances et du design déjà très éfficace, dans les deux domaines
des améliorations supplémentaires peuvent être apportées.

\subsection{\og Safe stack \fg}

\og Safe stack \fg se démarque des \og \gls{stackCookies} \fg également par
son approche, à la place d'essayer de détecter lorsqu'un dépassement de tampon à lieu,
les variables accèdées de manière dangereuse sont déplacées.
Cela permet d'annuler complétement l'éfficacité de ceux-ci, l'approche n'est plus de
freiner au maximum l'attaquant, mais de changer le paradigme afin de rendre impossible
ce type d'attaque.
Les canaris empêchent déjà le démarrage d'une attaque \gls{rop} depuis
la pile, cependant il est possible, sous certaines conditions, de réécrire correctement
le canari et donc de passer outre la protection. Un exemple simple est présenté dans
le \autoref{chap:attaque}.

% -----------------------------------------------------------------------------
\section{Évaluation des objectifs initiaux}

Les objectifs obligatoires du projet, tel que décrit dans le cahier des charges présent en
annexe, ont été remplis. Les deux concept \gls{cpi} et \gls{cps} ont été présentés
avec \og \gls{safeStack} \fg. Une description de \gls{llvm} ainsi que l'architecture
de l'implémentation de \og \gls{safeStack} \fg a été rédigée. Un historique des
méthodes de protection ainsi que certaines attaques liées fait office de vue
synoptique des attaques agissant sur le flot de contrôle, comme discuté lors des
rencontres. Deux exemples d'attaque permettant de tester les limites des canaris
ainsi que de \og \gls{safeStack} \fg ont été décrits et implémentés.

L'objectif optionnel d'analyser les coûts d'un portage de \gls{levee} sur la plateforme
ARM n'a pas été réalisé, bien que la plupart
des éléments nécessaires à cette analyse soit présent dans ce rapport, le temps
a manqué à produire un chapitre ou une section bien documentée concernant les
enjeux à prendre en compte.

\newpage

% -----------------------------------------------------------------------------
\section{Difficultés rencontrées}

La maîtrise du sujet traité --- bas niveau, sécurité des binaires, fonctionnement de
la mémoire, etc. --- a fait défaut au début du travail, une phase importante
de recherche et de compréhension des éléments sur lesquels se basent les
concepts décrits a été nécessaire. C'est aussi pourquoi les sections dédiées aux
récapitulatifs sont aussi fournies pour un rapport de ce type. Aujourd'hui je peux
dire que le sujet a bien été parcouru et certains aspects ont été
étudiés en profondeur. Les enjeux majeurs et les concepts les plus importants
ont été relayés dans ce rapport ainsi que les explications nécessaires à leur bonne
compréhension. La mise en place des attaques a été un long chemin parfois difficile.

% -----------------------------------------------------------------------------
\section{Sujets de recherche à développer}

On peut distinguer deux parties, \gls{cpi} et \og \gls{safeStack} \fg.
Pour \gls{cpi} les sujets de recherche à développer en premier serait l'intégration
de la technologique Intel MPX, le papier la mentionnait comme étant une technologie
à paraître, maintenant que celle-ci existe il serait intéressant de l'étudier de
plus prêt. Il serait aussi très intéressant de se relayé la partie thèorique
plus formelle, le \og proof of correctness \fg, en annexe du papier.
Pour \og \gls{safeStack} \fg, le sujet à développer plus en profondeur pourrait
être sa comparaison avec d'autres moyens permettant de se protéger des dépassements
de tampons.

De manière générale, un état des lieux plus précis de l'avancement ou non du
développement de \gls{cpi}/\gls{cps}, voir son intégration future au sein de \gls{llvm},
ainsi que les améliorations portées à \og \gls{safeStack} \fg par la communauté pourrait
être un bon moyen de se rendre compte du dynamisme et de la rapidité qu'à celle-ci à
développer des systèmes de sécurité.
