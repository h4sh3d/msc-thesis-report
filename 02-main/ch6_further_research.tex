\chapter{Further research}
\label{chap:furtherResearch}

It is possible to list an enormous amount of idea or further research in a field
like crypto-currencies or blockchain. But some of them more related to the work
done in this paper are listed in the following. Some of them are improvements
of the work already done but not yet ready for production, and some of them are
completely exploratory.

% -----------------------------------------------------------------------------
\section{Side-channel attack resistant implementation and improvements}

The proposed implementation into the library \texttt{secp256k1} rely on \texttt{libgmp}
for all complex mathematical calculus and \texttt{libgmp} is not strong against
side channel attacks, and it is normal, the library has not been developed for
that particular purpose. Therefore, a other implementation need to take the place
and handle, in constant time and constant memory if possible, the mathematical
calculus part. This is a big improvement that can be done, or must be done, before
hoping to use the module is some real case scenario.

\subsection{Second hash function}

The current implementation use the hash function \texttt{SHA256} implemented
into the library \texttt{secp256k1} for $\Pi$ and $\Pi'$. This is not complient
with the orignial paper requirements, a other hash function must be implemented
and used for $\Pi'$.

\subsection{Paillier cryptosystem}

Two major improvements or modifications could be performed specificaly on the
Paillier cryptosystem implementation. As shown in the original paper,
the Chinese Remainder Theorem can be used to optimize the decryption.
In the standard approach, with a private key
$(n, g, \lambda, \mu)$ and a ciphertext $c \in \mathbb{Z}_{n^2}^*$ it is possible
to compute the plaintext $m = L(c^{\lambda} \mod n^2) \cdot \mu \mod n$
where $L(x) = \frac{x-1}{n}$. With the CRT two function $L_p$ and $L_q$ are
define by

\begin{ceqn}
\begin{align*}
  L_p(x) = \frac{x-1}{p} \quad \text{and} \quad L_q(x) = \frac{x-1}{q}
\end{align*}
\end{ceqn}

Decryption can therefore be perform over mod $p$ and mod $q$ and recombining
modular residues afterwards:

\begin{ceqn}
\begin{align*}
  m_p = L_p(c^{p-1} \mod p^2) \ h_p \mod p \\
  m_q = L_q(c^{q-1} \mod p^2) \ h_q \mod q \\
  m = \text{CRT}(m_p, m_q) \mod pq
\end{align*}
\end{ceqn}

with precomputations

\begin{ceqn}
\begin{align*}
  h_p &= L_p(g^{p-1} \mod p^2)^{-1} \mod p \quad \text{and} \\
  h_q &= L_q(g^{q-1} \mod p^2)^{-1} \mod q
\end{align*}
\end{ceqn}

Paillier cryptosystem can be adapted to \gls{ec} cryptography as shown in the
paper \say{Trapdooring Discrete Logarithms on Elliptic Curves over Rings} by
Pascal Paillier \cite{10.1007/3-540-44448-3_44}. It is worth nothing however
that the curve construction is different than the curve used to sign and so
the code base cannot can not necessarily be reused.

\subsection{Zero-knowledge proofs}

Non-interactive zero-knowledge proofs are a big research field. The article
\say{From Extractable Collision Resistance to Succinct Non-interactive Arguments of Knowledge, and Back Again}
by Bitansky, Nir and Canetti, Ran and Chiesa, Alessandro and Tromer, and Eran
\cite{Bitansky:2012:ECR:2090236.2090263}
introduced the acronym zk-SNARK for zero-knowledge Succinct Non-interactive ARgument
of Knowledge that are the backbone of the Zcash protocol \cite{cryptoeprint:2014:349}.
In the recent paper \say{Bulletproofs: Efficient Range Proofs for Confidential Transactions}
\cite{cryptoeprint:2017:1066} a new non-interactive zero-knowledge proof
protocol with very short proofs and without a trusted setup is proposed.
Further research could be done to adapt the zero-knowledge proof construction
and migrate to a more generic approach, to remember that the zero-knowledge proof
construction proposed in the original paper dates from the early 2000s, progress
has been made since.


% -----------------------------------------------------------------------------
\section{Hardware wallets}

Hardware wallet devices have become increasingly popular with people and society. They
promise to keep the keys safe and, at least, expose less the keys thanks to a
dedicated and controlled environment. Thus, keys can be stored safly and, in an
organisation for exemple, multiple hardware wallets can be used to create a
multi-signature and control the funds.

The development of this threshold library, even if it is just a 2-out-of-2
multi-signature script equivalent, can be used to create real threshold hardware
wallet devices. Two hardware wallet devices can be setup together to create a
multi-user setup, or an hardware wallet device can be couple with a phone to secure
a lightweight wallet.

Usually, when a new Bitcoin wallet is created, a list of words called mnemonic is
shown to the user as a backup of his wallet key. The mnemonics are between twelve
and twenty-four and each word represent 11 bits of the primary seed \cite{Mnemonic},
for a threshold key it is not possible to represent all the data in the same way
given the size of the key (near 4.5 Kb). A other way to display and transmit these
information is needed to increase usability. Further research could be done to find
a better way to represent and display a threshold key.

The master tag is not included in the \texttt{DER} schema. Is the key itself responsible
to store this information or this information is a part of the setup and can be stored
outside, this question can be deepened.

% -----------------------------------------------------------------------------
\section{More generic threshold scheme}

% Implementing a more generic scheme as describe in paper \cite{10.1007/BFb0052253, 10.1007/978-3-642-27954-6_20}

\section{Schnorr signatures}

% Analyse Schnorr signatures for an application on payment channels
