% Terms
% -----
% format:  \newglossaryentry{<label>}{<settings>}
% example: \newglossaryentry{computer}
%{
%	name=computer,
%	description={is a programmable machine that receives input,
%		stores and manipulates data, and provides
%		output in a useful format}
%}

% \newglossaryentry{nosql}
% {
% 	name=NoSQL,
% 	description={Database not using the relational model and the \acrshort{sql} language}
% }


% Display ASLR
\newglossaryentry{aslr}
{
	name=ASLR,
	description={Address Space Layout Randomization, technique permettant de rendre aléatoire la position des ségments mémoires}
}

\newglossaryentry{cg-cfi}
{
	name=coarse-grained CFI,
	description={Coarse-grained CFI est une implémentation simplifiée du principe de Control-Flow integrity, échangeant sécurité contre de meilleures performances}
}

\newglossaryentry{fg-cfi}
{
	name=finest-grained CFI,
	description={Finest-grained CFI est une implémentation plus complète du principe de Control-Flow integrity garantissant une bonne sécurité mais ayant un coût élevé en performances}
}

\newglossaryentry{dep}
{
	name=DEP,
	description={Data Execution Prevention est une technique permettant de marquer un espace virtuel de mémoire comme étant non-exécutable grâce au NX bit}
}

\newglossaryentry{nx}
{
	name=NX,
	description={NX bit, pour No-eXecute bit, est une technique utilisée dans les processeurs pour dissocier les zones de mémoire contenant des instructions exécutables des zones contenant des données}
}

\newglossaryentry{stackCookies}
{
	name={stack cookies},
	description={Les stack cookies, ou stack canaries, sont des valeurs déposées sur la pile d'exécution après la valeur de retour lors de l'appel d'une fonction et son controlées à l'épilogue de la-dite fonction}
}

\newglossaryentry{stackCanaries}
{
	name={stack canaries},
	description={Les stack canaries sont un synonyme de stack cookies}
}

\newglossaryentry{levee}
{
	name={Levee},
	description={Levee est une implémentation des concepts de protection CPI, CPS et Safe Stack. Actuellement, en mai 2017, une partie du projet a été intégrée au sein de LLVM sous le nom de Safe Stack}
}

\newglossaryentry{llvm}
{
	name={LLVM},
	description={LLVM, à la base Low Level Virtual Machine et maintenant nom à part entière, est une collection d'outils permettant de construire des compilateurs}
}

\newglossaryentry{clang}
{
	name={Clang},
	description={Clang est un compilateur pour les langages de programmation \texttt{C/C++}, \texttt{Objective-C} et \texttt{Objective-C++}. Il utilise les bibliothèques que met à disposition LLVM}
}

\newglossaryentry{safeStack}
{
	name={safe stack},
	description={Safe Stack est un composant de de CPI/CPS permettant de traiter de manière particulière la gestion des pointeurs présents au sein de la pile d'exécution}
}

\newglossaryentry{lldb}
{
	name={LLDB},
	description={LLDB Debugger est présenté comme étant un débogueur dernière génération. Il fait parti des outils développés au sein du plus large projet qu'est LLVM}
}


% Acronyms
% --------
% format:  \newacronym{<label>}{<abbrv>}{<full>}
% example: \newacronym{lvm}{LVM}{Logical Volume Manager}
% plural:  \newacronym[longplural={Frames per Second}]{fpsLabel}{FPS}{Frame per Second}

% % Display Address Space Layout Randomization (ASLR)
% \newacronym{aslr}{ASLR}{Address Space Layout Randomization}


\newacronym{epfl}{EPFL}{École polytechnique fédérale de Lausanne}
\newacronym{nop}{NOP}{no operation}
\newacronym{cfg}{CFG}{control-flow graph}
\newacronym{cfi}{CFI}{control-flow integrity}
\newacronym{eof}{EOF}{end of file}
\newacronym{rop}{ROP}{return oriented programming}
\newacronym{cpi}{CPI}{code-pointer integrity}
\newacronym{cps}{CPS}{code-pointer separation}
\newacronym{gdb}{GDB}{the GNU project debugger}
\newacronym{pie}{PIE}{position-independent executable}
\newacronym{ast}{AST}{abstract syntax tree}
\newacronym{llvmIR}{LLVM IR}{LLVM intermediate representation}
